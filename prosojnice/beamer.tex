\documentclass{beamer}

\usepackage[utf8]{inputenc}
\usepackage[T1]{fontenc}
\usepackage[slovene]{babel}
\usepackage{lmodern}
\usepackage{array}
\usepackage{tikz}

\usetheme{Berlin}
\usecolortheme{default}
\useinnertheme[shadows]{rounded}
\useoutertheme{infolines}

\usepackage{palatino}
\usefonttheme{serif}

\newtheorem{definicija}{Definicija}
\newtheorem{izrek}{Izrek}

\begin{document}


% ===================================================================

\title{Priprava prosojnic v \LaTeX-u}
\subtitle{Uporaba paketa \texttt{beamer}}
\author{Ime Priimek}
\institute [FMF] {Fakulteta za matematiko in fiziko}
\date{}

\begin{frame}
\titlepage
\end{frame}

% -------------------------------------------------------------------
\begin{frame}
   \frametitle{Kratek pregled}
    \tableofcontents[pausesections]
\end{frame}
% ===================================================================

\section{Razporeditev vsebine}

% -------------------------------------------------------------------
\begin{frame}

\frametitle{Naštevanje}
   Za naštevanje lahko uporabimo okolje \texttt{itemize:}
\begin{itemize}
\item      Prva točka.
\item      Druga točka.
\item      Tretja točka.
\end{itemize}
   ali pa okolje \texttt{enumerate:}
\begin{enumerate}
\item      Prva točka.
\item      Druga točka.
\item      Tretja točka.
\end{enumerate}
\end{frame}
% -------------------------------------------------------------------
\begin{frame}
\frametitle   {Bloki z naslovom}
   Dele besedila lahko zapišemo v bloke.

   Uporabimo okolja \texttt{block, exampleblock, alertblock.}

   Za parameter okolja napišemo naslov bloka.
\begin{block}{Opomba}
      Tako je videti \texttt{block} z naslovom.
\end{block}
\begin{exampleblock}{Primer}
      Tako je videti \texttt{exampleblock} z naslovom.
\end{exampleblock}
\begin{alertblock}{Opozorilo}
      Tako je videti \texttt{alertblock} z naslovom.
\end{alertblock}
\end{frame}
% -------------------------------------------------------------------
\begin{frame}
\frametitle   {Bloki brez naslova}
   Blok lahko ima tudi prazen naslov.

   V takem primeru bo brez naslovne vrstice.
\begin{block}{}      Tako je videti \texttt{block} s praznim naslovom.
\end{block}
\begin{exampleblock}{}      Tako je videti \texttt{exampleblock} s praznim naslovom.
\end{exampleblock}
\begin{alertblock}{}
      Tako je videti \texttt{alertblock} s praznim naslovom.
\end{alertblock}
\end{frame}
% -------------------------------------------------------------------
\begin{frame}
\frametitle   {Stolpci}
   \begin{columns}
       \begin{column}{0.5 \textwidth}
       \begin{itemize}
\item            Besedilo lahko pišemo v več stolpcih.
\item            Osnovno okolje je columns.
\item             Posamezen stolpec opišemo v okolju column.
\item             Vsebina stolpca je lahko poljubna.
\item             Za primer imamo v desnem stolpcu napis v bloku in sliko sončnice.
         \end{itemize}
          \end{column}

         \begin{column}{0.5 \textwidth}
            \begin{exampleblock}{}
            \centering
            Slika v stolpcu.
            \end{exampleblock}
            \includegraphics{soncnica.jpg}
            \centering
          \end{column}
     \end{columns}
\end{frame}
% ===================================================================

\section{Matematične trditve}

% -------------------------------------------------------------------
\begin{frame}
\frametitle   {Praštevila}
     \begin{definicija}
        Praštevilo je naravno število, ki ima natanko dva delitelja.
     \end{definicija}
   \begin{exampleblock}{Zgledi}
      \begin{itemize}
\item         1 je praštevilo (ima samo enega delitelja: 1).
\item         2 je praštevilo (ima dva delitelja: 1 in 2).
\item         3 je praštevilo (ima dva delitelja: 1 in 3).
\item         4 ni praštevilo (ima tri delitelje: 1, 2 in 4).
\end{itemize}
\end{exampleblock}
\end{frame}
% -------------------------------------------------------------------
\begin{frame}
\frametitle   {Praštevila}
\begin{izrek}
      Praštevil je neskončno mnogo.
\end{izrek}
      \begin{proof}
Denimo, da je praštevil končno mnogo.
        \begin{itemize}
         \item Naj bo$ p$ največje praštevilo.
         \item Naj bo $q$ produkt števil $1, 2, \ldots, p.$
         \item Število $q+1$ ni deljivo z nobenim praštevilom, torej je $q+1$ praštevilo.
         \item To je protislovje, saj je $q+1>p.$
        \end{itemize}
      \end{proof}
\end{frame}
% ===================================================================

\section{Postopno odkrivanje vsebine}

% -------------------------------------------------------------------
\begin{frame}
 \frametitle  {Konstrukcija pravokotnice na premico p skozi točko T}
\begin{columns}
\begin{column}{0.5 \textwidth}
\begin{itemize}
\item <1->           Dani sta premica p in točka T.
\item <2->             Nariši lok k s središčem v T.
\item <3->             Premico p seče v točkah A in B.
\item <4->            Nariši lok m s središčem v A.
\item <5->            Nariši lok n s središčem v B in z enakim polmerom.
\item <6->             Loka se sečeta v točki C.
\item <7->             Premica skozi točki T in C je pravokotna na p.
\end{itemize}
\end{column}
\begin{column}{0.5 \textwidth}
\includegraphics<1>[height = 4cm]{pic1.png}
\includegraphics<2>[height = 4cm]{pic2.png}
\includegraphics<3>[height = 4cm]{pic3.png}
\includegraphics<4>[height = 4cm]{pic4.png}
\includegraphics<5>[height = 4cm]{pic5.png}
\includegraphics<6>[height = 4cm]{pic6.png}
\includegraphics<7>[height = 4cm]{pic7.png}
\end{column}
\end{columns}
\end{frame}
% -------------------------------------------------------------------

   Odkrivanje tabele po vrsticah
      Oznaka A B C D
      X 1 2 3 4
      Y 3 4 5 6
      Z 5 6 7 8

% -------------------------------------------------------------------

   Odkrivanje tabele po stolpcih
      Oznaka A B C D
      X 1 2 3 4
      Y 3 4 5 6
      Z 5 6 7 8

% ===================================================================

Razno

% -------------------------------------------------------------------

% ===================================================================

\end{document}